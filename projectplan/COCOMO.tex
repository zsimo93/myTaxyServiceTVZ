\section{Effort and Cost estimation - COCOMO}
In this section effort and costs of the project will be evaluated, in terms of month-person units. A detailed analysis of the scale and cost drivers will follow in the upcoming paragraphs, as defined in the COCOMO II estimation standard.\\

\subsection{Scale Drivers}
    \begin{itemize}
        \item \textbf{Precedentedness [PREC]}: This is the first Enterprise project developed by our team. However the team members are quite familiar with the Java language and basic language constructs and patterns. The level is therefore \textbf{low}.
        
        \item \textbf{Development Flexibility [FLEX]}: The customer only set general goals, without restricting the flow to specific processes. However, at an high level, the taxi reservation process is quite standardized, which therefore means the level is \textbf{high}.
        
        \item \textbf{Architecture/risk resolution [RESL]}: 
        The architecture/risk resolution requires a more in depth analysis of some specific points, in particular:
        \begin{itemize}
            \item The Risk Management Plan generally identifies critical risk items and establishes milestones for resolving them. Rating Level: Nominal.
            \item Schedule and budget have been set, but at some level of abstraction. Rating Level: Nominal.
            \item The percent of development schedule spent on the architecture is at a Nominal rating.
            \item Percent of required top software architects available: High.
            \item Tool support available to resolve risk items: Nominal.
            \item Level of uncertainty in key architecture drivers: High
            \item Number and criticality of risk items: 2-4 => Nominal.
            \\Overall rating: \textbf{High}
        \end{itemize}
        
        \item \textbf{Team cohesion [TEAM]}: The development team works really well and in harmony. Developers knew each other years before starting the project and there are practically no communication problems between them. So the level is \textbf{extra high}.
        
        \item \textbf{Process maturity [PMAT]}: Given the calculation KPAs, by doing an arithmetic approximation, the result of the EPML is 3: \textbf{High}.
        % KPA: 3 Always + 1 Frequently + 1 Half + 3 Rarely || + 5 Not Apply + 5 Don't Know
        % EPML: 5 x (3*100% + 1*75% + 1*50% + 3*1% ) / (100 * 8) = 2.675
        % PMAT Rating: 2: Nominal; 3: High
        
    \end{itemize}
    
\subsection{Cost Drivers}
    \begin{enumerate}
        \item \textbf{Required Software Reliability [RELY]}: Since the software simply replaces a manual procedure, no particular losses would be generated by a system failure. Rating Level: \textbf{Low}.
        
        \item \textbf{Data Base Size [DATA]}: Test data is considered to don't be particularly huge. Rating Level: \textbf{Nominal}.
        
        \item \textbf{Product Complexity [CPLX]}:
        The product complexity can be divided in some more specific topics. An overview of the specific ratings is presented below:
        \begin{itemize}
            \item Control Operations: High.
            \item Computational Operations: High.
            \item Device-dependent Operations: Nominal.
            \item Data Management Operations: High.
            \item User Interface Management Operations: Nominal.
        \end{itemize}
            
        So the average rating level is \textbf{High}.
        
        \item \textbf{Developed for Reusability [RUSE]}: The project development will be done with the idea of reusable components to the level of the project (there are no plans of other programs to develop which will interact with it). Rating level: \textbf{Nominal}.
        
        \item \textbf{Documentation Match to Life-Cycle Needs [DOCU]}: The most life-cycle needs are covered by previous documents. However some parts still need some more documentation. Rating Level: \textbf{Low}.
        
        \item \textbf{Execution Time Constraint [TIME]}: There are no particular constraints about the execution times, but still the application must respond quickly to user inputs. Rating Level: \textbf{High}.
        
        \item \textbf{Main Storage Constraint [STOR]}: Since only the database must be stored, a minimum part of the available storage will be used. Rating Level: \textbf{Nominal}.
        
        \item \textbf{Platform Volatility [PVOL]}: Most of the technologies are really stable and there is no perception of significant changes in less than one year. Rating Level: \textbf{Low}.
        
        \item \textbf{Analyst Capability [ACAP]}: The analysts are quite efficient in extracting the needed design and in cooperating to find the best architectural solutions. Rating Level: \textbf{High}.
        
        \item \textbf{Programmer Capability [PCAP]}: The programmers as a team ave a really effective communication and can cooperate really well. Rating Level: \textbf{Very High}.
        
        \item \textbf{Personnel Continuity [PCON]}: There are no information about the statistical personnel turnover. Rating Level: \textbf{Nominal}.
        
        \item \textbf{Application Experience [APEX]}: The team is composed by only beginners software engineers. Therefore there is practically no experience in the development of this kind of products at an enterprise level. Rating Level \textbf{Very Low}.
        
        \item \textbf{Platform Experience [PLEX]}: All team members have quite a basic comprehension and few experience of the chosen platform. Rating Level: \textbf{Low}.
        
        \item \textbf{Language and Tool Experience [LTEX]}: All team members are quite experienced with the Java language and the common tools usually used for developing programs in that context. Rating Level: \textbf{High}.
        
        \item \textbf{Use of Software Tools [TOOL]}: Given the semi-mature definition of the application life cycle, and the choice of a continuous integration tool, an effective tool set has been defined. Rating Level: \textbf{High}.
        
        \item \textbf{Multisite Development [SITE]}: The development is fully collocated. Rating Level: \textbf{Extra High}
        
        \item \textbf{Required Development Schedule [SCED]}: The development schedule is properly set without any particular stretch-outs or compressions. Rating Level: \textbf{Nominal}.
        
    \end{enumerate}
    
    \subsection{Summary}
    Given the effort ($\epsilon$) equation, which is:
    \begin{equation}
        \epsilon = A \times Size^E \times EAF
    \end{equation}
    
    Where:
    \begin{itemize}
        \item A = 2.94 as in COCOMO II.2000
        \item Size: actual size of the product to be developed in terms of KSLOC (thousands of Source Lines Of Code).
        \begin{equation}
            Size = \frac{UFP \times LF}{1000}
        \end{equation}
        Where LF = 46 SLOCs as for the avarage value of J2EE applications.
        
        \item E : Exponent derived from the five Scale Drivers, with
        \begin{equation}
            E = B + 0.01 \times \sum_{j=1}^5 SF_j
        \end{equation}
        With B = 0.91 as in COCOMO II.2000
        \item EAF : Effort Adjustment Factor derived from the Cost Drivers, with
        \begin{equation}
            EAF = \prod_{i=1}^N EM_i
        \end{equation}
    \end{itemize}
    
    
 {\renewcommand{\arraystretch}{1.5}

\begin{tabularx}{\textwidth}{X  X r}
    \hline 
    \textbf{Scale Factors} & \textbf{Rating Level} &\textbf{SF Value}\\ 
    \hline 
    PREC & Low & 4.96\\
    \hline 
    FLEX & High & 2.03\\
    \hline
    RESL & High & 2.83\\
    \hline
    TEAM & Extra High & 0.00\\
    \hline
    PMAT & Nominal & 4.68\\
    \hline
    \textbf{TOTAL (Sum)} &  & \textbf{14.5}\\
    \hline
    \caption{Scale Drivers Recap}\label{tab:sf-summary}\\
\end{tabularx}}


{\renewcommand{\arraystretch}{1.5}

\begin{tabularx}{\textwidth}{X  X r}
    \hline 
    \textbf{Cost Drivers} & \textbf{Rating Level} &\textbf{Effort Multiplier}\\ 
    \hline 
    RELY & Low & 0.92\\
    \hline 
    DATA & Nominal & 1.00\\
    \hline
    CPLX & High & 1.17\\
    \hline
    RUSE & Nominal & 1.00\\
    \hline
    DOCU & Low & 0.91\\
    \hline
    TIME & High & 1.11\\
    \hline
    STOR & Nominal & 1.00\\
    \hline
    PVOL & Low & 0.87\\
    \hline
    ACAP & High & 0.85\\
    \hline
    PCAP & Very High & 0.76\\
    \hline
    PCON & Nominal & 1.00\\
    \hline
    APEX & Very Low & 1.22\\
    \hline
    PLEX & Low & 1.09\\
    \hline
    LTEX & High & 0.91\\
    \hline
    TOOL & High & 0.90\\
    \hline
    SITE & Extra High & 0.80\\
    \hline
    SCED & Nominal & 1.00\\
    \hline
    \textbf{TOTAL (Product)} &  & \textbf{0.53}\\
    \hline
    \caption{Cost Drivers Recap}\label{tab:em-summary}\\
\end{tabularx}}

Therefore the estimated effort is:

\begin{equation}
        \epsilon = 2.94 \times (\frac{126\times 46}{1000})^{0.91+0.01\times 14.5} \times 0.53 = \textbf{9.95 person-month}
\end{equation}

with an estimated duration of:
\begin{equation}
        \delta = C \times \epsilon^{D+0.2\times(E-B)} = 
        3.67 \times 9.95^{0.28+0.2\times 0.01 \times 14.5} = 
        \textbf{7.5 months}
\end{equation}
\textit{Note: C = 3.67 and D = 0.28 as for COCOMO II.2000}\\\\
% C=3.67 | D = 0.28
And a (minimum) required number of people of:
\begin{equation}
        \pi = \lceil \epsilon / \delta \rceil = \textbf{2 persons}
\end{equation}