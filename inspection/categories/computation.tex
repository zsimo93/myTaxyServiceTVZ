\begin{itemize}
\item[44.] "The term \textit{brute force} refers to solutions where someone applies an obvious but excessive technique when more refined and effective alternatives are available. ". \\
Given this definition, the analyzed code has generated many concerns.\\
From a very basic analysis the code looks not so bad, because it strongly relies on framework methods and therefore avoids to redefine methods to handle already solved problems.\\
However some design problmes will be discussed later. 
\item[45.] Computations order is correct, so is parenthesizing.
\item[46.] There are really few calculations. So there are no operator precedence issues, neither divisions, nor integer arithmetic in general is used.
\item [49.] Boolean operators are always used in either \textit{if} or \textit{for} conditions. In most cases comparison operator are used, such as == or != to check if an extracted node is whether null or not. \textit{Lines 1322,1329,1335,1340,1347,1350,1353.}\\ Other comparison operators, such as $<$, are used in \textit{for} statements as termination conditions to scan the whole collection in object. \textit{Lines 1342,1408,1578}.\\Other Boolean operators, such as \&\&, are used in \textit{lines 1350,1353,1365,1368,1587}.\\ All the operators are used in the correct way and with a correct behaviour.
\item [50.] At \textit{line 1630} in the catch field a generic SAXException is thrown and the real exception is hidden. This is not good practice.
\item [51.] At \textit{lines 1354,1369} String object are automatically converted to Object and passed as parameters to the put method.

\end{itemize}
